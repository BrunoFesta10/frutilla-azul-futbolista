\documentclass[10pt,a4paper]{article}
\usepackage{ tipa }
\input{AEDmacros}
\usepackage{caratula} % Version modificada para usar las macros de algo1 de ~> https://github.com/bcardiff/dc-tex


\titulo{Trabajo practico 1}

\fecha{\today}

\materia{Algoritmos y estructuras de datos}
\grupo{Grupo: LEQNFVIO}

\integrante{Festa, Bruno Alejandro}{690/23}{brunofesta2004@gmail.com}
\integrante{Tocto, Anthony Armando}{342/22}{armando.ddvv@gmail.com}
\integrante{Barg Oyanguren, Ciro}{597/23}{cirobargoyanguren@gmail.com}
\integrante{Zapata, Juan Francisco}{1191/23}{juanfzapata123@gmail.com}
% Pongan cuantos integrantes quieran

% Declaramos donde van a estar las figuras
% No es obligatorio, pero suele ser comodo
\graphicspath{{../static/}}

\begin{document}

\maketitle

%\section{Ejemplo de sección}
%\subsection{Subsección: ambientes comunes de \LaTeX}

%Lo principal: las fórmulas. Se puede poner en una linea, como $x_i = x_{i-1} + x_{i-2}$.
%\newline
% \hfill \break

\section{Ejercicio 1}
\begin{proc}{redistibucionDeLosFrutos}{\In recursos:\TLista{\float}, \In cooperan:\TLista{\bool}}{\Tlista{\float}{}}
    \requiere{\longitud {cooperan} = \longitud {recursos}}
    \requiere{(\forall  i: \ent)(0 \leq i < \longitud {recursos} \implicaLuego recursos[i] \geq 0}
    \asegura{(\exists L : \TLista{\float})(esListaDeRecursosAntesDelFondoComun(recursos,cooperan,L)\yLuego res = L)}
    \vspace{5mm}
    \pred{esDisribucionDeRecursos}{cooperan : \TLista{bool}, recursos : \TLista{\float}, L : \TLista{\float}}{(\forall i : \ent)(0 \leq i < \longitud L \implicaLuego \IfThenElse{cooperan[i]= \True}{L[i]=\frac{fondoComun(recursos,cooperan)}{\longitud {recursos}}\\}{L[i]=\frac{fondoComun(recursos,cooperan)}{\longitud {recursos}}+recursos[i])}}
    \aux{fondoComun}{In recursos :\Tlista{\float}, In cooperan : \Tlista{\bool}}{\float}{\sum\limits_{i=0}^{|recursos|-1} \IfThenElse{(cooperan[i]=\True)}{recursos[i]}{0}}
    
\end{proc}






\section{Ejercicio 2}
\begin{proc}{trayectoriaDeLosFrutosIndividualesALargoPlazo}{\Inout  trayectoria : \Tlista{\TLista{\float}}, \In cooperan:\TLista{\bool},\\ \In apuestas : \TLista{\TLista{\float}}, \In pagos : \TLista{\TLista{\float}}, \In eventos :\TLista{\TLista{\float}}}

\vspace{5mm}
    \requiere{\longitud {cooperan} = \longitud {trayectoria} = \longitud {eventos} = \longitud {pagos} = \longitud {apuestas}}
      \vspace{5mm} 
    \requiere{(\forall  i: \ent)(0 \leq i < \longitud {apuestas} \implicaLuego \sum\limits_{k=0}^{\longitud {apuestas[i]}} \text{apuestas[i][k] = 1)}}
    \vspace{5mm} 
    \requiere{(\forall i,e : \ent)(0 \leq i < \longitud {pagos} \land 0 \leq e < \longitud{pagos[i]} \implicaLuego pagos[i][e]>0)}
      \vspace{5mm} 
    \requiere{(\forall i,e : \ent)(0 \leq i < \longitud {trayectoria} \land 0 \leq e < \longitud{trayectoria[i]} \implicaLuego trayectoria[i][e]>0)}
      \vspace{5mm} 
    \asegura{(\forall individuo,nroDeEvento:\ent)(0 \leq individuo < \longitud {recursos}\land 0 < nroDeEvento\\ <\longitud {cooperan}\implicaLuego(\exists Lista : \TLista{\float})(esListaDespuesDeDistribuirElFondoComun(individuo,nroDeEvento,pagos,apuestas,Lista\\,trayectoria,cooperan)\land trayectoria[individuo][nroDeEvento]=Lista[individuo]))}
    \vspace{5mm} 
    \pred{esListaDespuesDeDistribuirElFondoComun}{individuo : \ent, nroDeEvento : \ent, pagos : \TLista{\TLista{\float}}, apuesta \TLista{\TLista{\float}}, \\listaDistribuida:\TLista{\float}, trayectoria\TLista{\TLista{\float}}, cooperan \TLista{\bool}} {(\exists listaDeRecursosActualizada : \TLista{\float})(esListaDeRecursosActualizada(listaDeRecursosActualizada,individuo,\\nroDeEvento,trayectoria,pagos,apuestas)\yLuego esListaDeRecursosAntesDelFondoComun(cooperan,listaDeRecursosActualizada\\,listaDistribuida)}
    \vspace{5mm} 
    \pred{esListaDeRecursosActualizada}{lista : \TLista{\float}, individuo : \ent, nroDeEvento : \ent, trayectoria:\Tlista{\TLista{\float}}, pagos : \TLista{\TLista{\float}}, apuesta \TLista{\TLista{\float}}}{lista[individuo]=calculoIndividual(individuo,nroDeEvento,trayectoria,pagos,apuestas)}
    \vspace{5mm} 
    \aux{calculoIndividual}{individuo : \ent, nroDeEvento : \ent, trayectoria : \TLista{\TLista{\float}}, pagos : \TLista{\TLista{\float}}, apuestas : \TLista{\TLista{\float}}}{\float}{trayectoria[individuo][nroDeEvento-1]*pagos[individuo][nroDeEvento]*\\apuestas[individuo][nroDeEvento]}
     \vspace{5mm} 
    \pred{esDisribucionDeRecursos}{cooperan : \TLista{bool}, recursos : \TLista{\float}, L : \TLista{\float}}{(\forall i : \ent)(0 \leq i < \longitud L \implicaLuego \IfThenElse{cooperan[i]= \True}{L[i]=\frac{fondoComun(recursos,cooperan)}{\longitud {recursos}}\\}{L[i]=\frac{fondoComun(recursos,cooperan)}{\longitud {recursos}}+recursos[i])}}
     \vspace{5mm} 
    \aux{fondoComun}{In recursos :\Tlista{\float}, In cooperan : \Tlista{\bool}}{\float}{\sum\limits_{i=0}^{|recursos|-1} \IfThenElse{(cooperan[i]=\True)}{recursos[i]}{0}}
 
\end{proc}












\section{Ejercicio 3}
\begin{proc}{trayectoriaExtrañaEscalera}{\In trayectoria: \TLista{\float}}{\bool{}}
	\requiere{\longitud T \geq 2}
    \requiere{res=\False}
	\asegura{Res=\True \iff  (\exists j : \ent)((0\leq j < \longitud T) \yLuego 
    ((\longitud T \leq 2) \land Distintos(T)) \lor \\ 
    ((j=0)\land{primerElemEsMaximo(T) \land esUnicoMaximoLocal(T,j)}) \lor \\ 
    ((j=\longitud T -1)\land{ultimoElemEsMaximo(T) \land esUnicoMaximoLocal(T,j)})\lor \\ 
    ({esMaximoLocal(T,j) \land esUnicoMaximoLocal(T,j)))}}
    
	\pred{primerElemEsMaximo}{T:\TLista{\float}}{T[0]> T[1]} 
    \pred{ultimoElemEsMaximo}{T:\TLista{\float}}{T[\longitud T -1]> T[\longitud T -2]} 
    \pred{esMaximoLocal}{T:\TLista{\float}, j:\ent}{T[j-1]<T[j] \land T[j]>T[j+1]}
    \pred{esUnicoMaximoLocal}{T:\TLista{\float}, j:\ent}{\neg (\exists i : \ent)((1\leq i < \longitud T -1) {\yLuego} (i \neq j)\land esMaximoLocal(T,i))}
    \pred{distintos}{T:\TLista{\float}}{T[0]\neq T[1]}
\end{proc}









\section{Ejercicio 4}
\begin{proc}{individuoDecideSiCooperarONo}{\In individuo : \nat, \In recursos : \TLista{\float},\Inout cooperan : \TLista{\bool}, \In apuestas : \TLista{\float}, \In pagos : \TLista{\TLista{\float}}, \In eventos : \TLista{\TLista{\float}}}{}
\vspace{5mm}
\requiere{(\forall  i: \ent)(0 \leq i < \longitud {apuestas} \implicaLuego \sum\limits_{k=0}^{\longitud {apuestas[i]}} \text{apuestas[i][k] = 1)}}
\vspace{5mm}
\requiere{\longitud {cooperan} = \longitud {recursos} = \longitud {eventos} = \longitud {pagos} = \longitud {apuestas}}
\vspace{5mm}
\requiere{0 \leq individuo < \longitud{recursos}}
\vspace{5mm}
\requiere{(\forall i,e : \ent)(0 \leq i < \longitud {pagos} \land 0 \leq e < \longitud{pagos[i]} \implicaLuego pagos[i][e]>0)}
\vspace{5mm}
\requiere{(\forall  i: \ent)(0 \leq i < \longitud {recursos} \implicaLuego recursos[i] \geq 0}
\vspace{5mm}
\asegura{(\forall i : \ent)(0 \leq i < \longitud{recursos} \implicaLuego (\exists listaFinalDeRecursosSiCoopera: \TLista{\float})\\(esListaDespuesDeDistribuirElFondoComun(i,\longitud{eventos}-1,pagos,apuestas,listaFinalDeRecursosSiCoopera,\\ setAt(cooperan,individuo,\True))}

\vspace{3mm}
\asegura{(\forall i : \ent)(0 \leq i < \longitud{recursos} \implicaLuego (\exists listaFinalDeRecursosSiNOCoopera: \TLista{\float})\\(esListaDespuesDeDistribuirElFondoComun(i,\longitud{eventos}-1,pagos,apuestas,listaFinalDeRecursosSiNOCoopera,\\ setAt(cooperan,individuo,\False))}
\vspace{3mm}
\vspace{3mm}
\asegura{\IfThenElse{(listaFinalDeRecursosSiCoopera[individuo] \leq listaFinalDeRecursosSiNOCoopera[individuo])\\}{(cooperan[individuo]=\False)}{(cooperan[individuo]=\True)}))}
\vspace{5mm}
\pred{esListaDespuesDeDistribuirElFondoComun}{individuo : \ent, nroDeEvento : \ent, pagos : \TLista{\TLista{\float}}, apuesta \TLista{\TLista{\float}}, listaDistribuida:\TLista{\float}, trayectoria\TLista{\TLista{\float}}, cooperan \TLista{\bool}} {(\exists listaDeRecursosActualizada : \TLista{\float})(esListaDeRecursosActualizada(listaDeRecursosActualizada,individuo,\\nroDeEvento,trayectoria,pagos,apuestas)\\ \yLuego esListaDeRecursosAntesDelFondoComun(cooperan,listaDeRecursosActualizada\\,listaDistribuida)}
\vspace{5mm}
\pred{esDisribucionDeRecursos}{cooperan : \TLista{bool}, recursos : \TLista{\float}, L : \TLista{\float}}{(\forall i : \ent)(0 \leq i < \longitud L \implicaLuego \IfThenElse{cooperan[i]= \True}{L[i]=\frac{fondoComun(recursos,cooperan)}{\longitud {recursos}}\\}{L[i]=\frac{fondoComun(recursos,cooperan)}{\longitud {recursos}}+recursos[i])}}
\vspace{5mm}
 \aux{fondoComun}{In recursos :\Tlista{\float}, In cooperan : \Tlista{\bool}}{\float}{\sum\limits_{i=0}^{|recursos|-1} \IfThenElse{(cooperan[i]=\True)}{recursos[i]}{0}}

 \vspace{5mm} 
    \pred{esListaDeRecursosActualizada}{lista : \TLista{\float}, individuo : \ent, nroDeEvento : \ent, trayectoria:\Tlista{\TLista{\float}}, pagos : \TLista{\TLista{\float}}, apuesta \TLista{\TLista{\float}}}{lista[individuo]=calculoIndividual(individuo,nroDeEvento,trayectoria,pagos,apuestas)}

\end{proc}







\section{Ejercicio 5}
\begin{proc}{individuoActualizaSuApuesta}{\In individuo : \nat, \In recursos : \TLista{\float},\In cooperan : \TLista{\bool}, \Inout apuestas : \TLista{\float}, \In pagos : \TLista{\TLista{\float}}, \In eventos : \TLista{\TLista{\float}}}{}
\requiere{(\forall  i: \ent)(0 \leq i < \longitud {apuestas} \implicaLuego \sum\limits_{k=0}^{\longitud {apuestas[i]}} \text{apuestas[i][k] = 1)}}
\vspace{5mm}
\requiere{\longitud {cooperan} = \longitud {recursos} = \longitud {eventos} = \longitud {pagos} = \longitud {apuestas}}
\vspace{5mm}
\requiere{0 \leq individuo < \longitud{recursos}}
\vspace{5mm}
\requiere{(\forall i,e : \ent)(0 \leq i < \longitud {pagos} \land 0 \leq e < \longitud{pagos[i]} \implicaLuego pagos[i][e]>0)}
\vspace{5mm}
\requiere{(\forall  i: \ent)(0 \leq i < \longitud {recursos} \implicaLuego recursos[i] \geq 0}
\vspace{5mm}
\asegura{(\exists listaDeRecursosQueMaximizaGananciasDeIndividuo,mejorApuesta : \TLista{\float})\\(esApuestaValida(mejorApuesta,individuo) \yLuego esListaDeApuestas(i,\longitud{eventos},\\pagos,setAt(apuestas,individuo,mejorApuesta),listaDeRecursosQueMaximizaGananciasDeIndividuo,cooperan))\\ \yLuego \\ \neg(\exists ListaDeRecursosCualquiera,apuestaCualquiera : \TLista{\float})(esApuestaValida(apuestaCualquiera,individuo) \yLuego \\ esMejorApuesta(apuestaCualquiera,mejorApuesta,individuo,eventos,pagos,cooperan)))}
\vspace{5mm}
\pred{esListaDeApuestas}{individuo : \ent, nroDeEvento : \ent, pagos : \TLista{\TLista{\float}}, apuesta \TLista{\TLista{\float}}, \\listaDistribuida:\TLista{\float}, trayectoria\TLista{\TLista{\float}}, cooperan \TLista{\bool}}{(\forall i : \ent)(0 \leq i < \longitud{recursos} \implicaLuego \\esListaDespuesDeDistribuirElFondoComun(i,\longitud{eventos},\\pagos,setAt(apuestas,individuo,mejorApuesta),listaDeRecursosQueMaximizaGananciasDeIndividuo,cooperan)}
\vspace{5mm}
\pred{esApuestaValida}{apuestaAVerificar : \TLista{\float}}{\sum\limits_{i=0}^{\longitud {apuestaAVerificar}-1} \text{apuestaAVerificar[i]}=1 \land \longitud{apuestaAVerificar}= \longitud{apuestas[individuo]}}
\vspace{5mm}
\pred{esMejorApuesta}{apuesta1 : \TLista{\float}, apuesta2 : \TLista{\float}, individuo : \nat, eventos : \TLista{\TLista{\float}}, pagos : \TLista{\TLista{\float}}, cooperan : \TLista{\bool}}{(\exists ListaDeRecursosApuesta1,ListaDeRecursosApuesta2 : \TLista{\float})(esListaDespuesDeDistribuirElFondoComun\\(individuo,\longitud{eventos},pagos,apuesta1,ListaDeRecursosApuesta1,cooperan) \\ \land (esListaDespuesDeDistribuirElFondoComun(individuo,\longitud{eventos},pagos,apuesta2\\,ListaDeRecursosApuesta2,cooperan) \land \\ListaDeRecursosApuesta1[individuo] \geq ListaDeRecursosApuesta2[individuo] )}
\vspace{5mm} 
\pred{esListaDespuesDeDistribuirElFondoComun}{individuo : \ent, nroDeEvento : \ent, pagos : \TLista{\TLista{\float}}, apuesta \TLista{\TLista{\float}}, \\listaDistribuida:\TLista{\float}, trayectoria\TLista{\TLista{\float}}, cooperan \TLista{\bool}} {(\exists listaDeRecursosActualizada : \TLista{\float})(esListaDeRecursosActualizada(listaDeRecursosActualizada,individuo,\\nroDeEvento,trayectoria,pagos,apuestas)\yLuego esListaDeRecursosAntesDelFondoComun\\(cooperan,listaDeRecursosActualizada,listaDistribuida)}
\vspace{5mm} 
\pred{esListaDeRecursosActualizada}{lista : \TLista{\float}, individuo : \ent, nroDeEvento : \ent, trayectoria:\Tlista{\TLista{\float}}, pagos : \TLista{\TLista{\float}}, apuesta \TLista{\TLista{\float}}}{lista[individuo]=calculoIndividual(individuo,nroDeEvento,trayectoria,pagos,apuestas)}
\vspace{5mm} 
\aux{calculoIndividual}{individuo : \ent, nroDeEvento : \ent, trayectoria : \TLista{\TLista{\float}}, pagos : \TLista{\TLista{\float}}, apuestas : \TLista{\TLista{\float}}}{\float}{trayectoria[individuo][nroDeEvento-1]*pagos[individuo][nroDeEvento]*\\apuestas[individuo][nroDeEvento]}   
\vspace{5mm}
\pred{esDisribucionDeRecursos}{cooperan : \TLista{bool}, recursos : \TLista{\float}, L : \TLista{\float}}{(\forall i : \ent)(0 \leq i < \longitud L \implicaLuego \IfThenElse{cooperan[i]= \True}{L[i]=\frac{fondoComun(recursos,cooperan)}{\longitud {recursos}}\\}{L[i]=\frac{fondoComun(recursos,cooperan)}{\longitud {recursos}}+recursos[i])}}
\vspace{5mm}
\aux{fondoComun}{In recursos :\Tlista{\float}, In cooperan : \Tlista{\bool}}{\float}{\sum\limits_{i=0}^{|recursos|-1} \IfThenElse{(cooperan[i]=\True)}{recursos[i]}{0}}
\end{proc}


\section{Ejercicio 6}


\begin{itemize}
    \item $P_c \equiv res = recursos  \land  i = 0 \land 0\leq recursos\\ $
     \item $Q_c \equiv res = recurso(apuesta_c * pago_c)^{apariciones(eventos, T)}*(apuesta_s * pagos_s)^{apariciones(eventos, F)} \\$
     \item $I \equiv 0 \leq i \leq |eventos|$ \land   res = {\prod_{j=0}^{i-1} \text{if (eventos[j]) then ($apuesta_c * pago_c$) else ($apuesta_s * pago_s$) fi} }$\\
     \item $B \equiv i < |eventos|$
     \\
     \item  $fv = |eventos| - i$
     \\ 
     \\
    

     \item $\{\neg B \land I\} \implica Q_c$

$\neg (i  < |eventos|) \land 0 \leq i \leq |eventos|$ \land   res = {\prod_{j=0}^{i-1} \text{if (eventos[j]) then ($apuesta_c * pago_c$) else ($apuesta_s * pago_s$) fi} }$\equiv \\
|eventos|\leq i \land i\leq |eventos| \land res = {\prod_{j=0}^{|eventos|-1} \text{if (eventos[j]) then ($apuesta_c * pago_c$) else ($apuesta_s * pago_s$) fi} }\equiv\\res = recurso(apuesta_c * pago_c)^{apariciones(eventos, T)}*(apuesta_s * pagos_s)^{apariciones(eventos, F)} \equiv Q_c\\

    \item $P_c \implica I$\\
     Como por $P_c$ se tiene que i=0 en el invariante queda que\\ $res = {\prod_{j=0}^{-1} \text{if (eventos[j]) then ($apuesta_c * pago_c$) else ($apuesta_s * pago_s$) fi}=0 }$ \\ por otra parte como por $P_c$ $res=recursos $ y $0\leq recursos$ se tiene que $0\leq res$ queda que $res=0$ es una Tautologia\\ \\ \\ \\ \\ \\
    \item $\{B \land I\} S \{I\} \equiv (B \land  I) \implica wp(S, I)$\\
    Donde S\equiv if\hspace{0.1cm} $eventos[i]$ \hspace{0.1cm}then\\
    S_1 \hspace{1cm} res := res*apuesta.c*pago.c\\ else\\S_2  \hspace{1cm} $res := res*apuesta.s*pago.s$\\endif\\S_3 
  \hspace{1cm} $i := i+1$ 
    
    $wp(S,I)\equiv wp(if\hspace{0.1cm} eventos[i]\hspace{0.1cm} Then\hspace{0.1cm} res = res*apuesta.c*pago.c\hspace{0.1cm} else\hspace{0.1cm} res*apuesta.s*pago.s, wp(i:= i + 1, I))$ \\$wp(S_3,I)\equiv def(i+1)\yLuego 0 \leq i+1 \leq |eventos| \land   res = {\prod_{j=0}^{i+1-1} \text{if (eventos[j]) then ($apuesta_c * pago_c$) else ($apuesta_s * pago_s$) fi} }$\\ wp(S,I)\equiv def(eventos[i])\yLuego ((eventos[i]=True\land wp(S_1, wp(S_3, I)) ) \lor (eventos[i]=False \land wp(S_2, wp(S_3, I))))\\wp(S,I)\equiv $0\leq i <|eventos|$\yLuego ((eventos[i]=True\land wp(S_1, wp(S_3, I)) ) \lor (eventos[i]=False \land wp(S_2, wp(S_3, I)))) \\ \\wp(S_1,wp(S_3,I))\equiv def(res*apuesta.c*pago.c)\yLuego $0 \leq i+1 \leq |eventos| \land   \\res*apuesta.c*pago.c = {\prod_{j=0}^{i} \text{if (eventos[j]) then ($apuesta_c * pago_c$) else ($apuesta_s * pago_s$) fi} }$ \equiv \\$0 \leq i+1 \leq |eventos| \land   \\res*apuesta.c*pago.c = {\prod_{j=0}^{i-1} \text{if (eventos[j]) then ($apuesta_c * pago_c$) else ($apuesta_s * pago_s$) fi} }$$*(apuesta.c*pago.c)$ \equiv \\ $0 \leq i < |eventos| \land   \\res= {\prod_{j=0}^{i-1} \text{if (eventos[j]) then ($apuesta_c * pago_c$) else ($apuesta_s * pago_s$) fi} }$       \\ \\ wp(S_2,wp(S_3,I))\equiv def(res*apuesta.s*pago.s)\yLuego $0 \leq i+1 \leq |eventos| \land   \\res*apuesta.s*pago.s = {\prod_{j=0}^{i} \text{if (eventos[j]) then ($apuesta_c * pago_c$) else ($apuesta_s * pago_s$) fi} }$ \equiv \\$0 \leq i+1 \leq |eventos| \land   \\res*apuesta.s*pago.s = {\prod_{j=0}^{i-1} \text{if (eventos[j]) then ($apuesta_c * pago_c$) else ($apuesta_s * pago_s$) fi} }$$*(apuesta.s*pago.s)$ \equiv \\ $0 \leq i<|eventos| \land   \\res= {\prod_{j=0}^{i-1} \text{if (eventos[j]) then ($apuesta_c * pago_c$) else ($apuesta_s * pago_s$) fi} }$
    \\ \\
    Ahora calculemos $I \land B$ y veamos si implica lo de arriba: \\
    $I \land B \equiv \\ 0 \leq i < |eventos|$ \land   res = {\prod_{j=0}^{i-1} \text{if (eventos[j]) then ($apuesta_c * pago_c$) else ($apuesta_s * pago_s$) fi} } \equiv \\
    0 \leq i < |eventos|$ \land   
    ((eventos[i]=True \land res = {\prod_{j=0}^{i-1} \text{if (eventos[j]) then ($apuesta_c * pago_c$) else ($apuesta_s * pago_s$) fi} })  \\
     \lor (eventos[i]=False \land res = {\prod_{j=0}^{i-1} \text{if (eventos[j]) then ($apuesta_c * pago_c$) else ($apuesta_s * pago_s$) fi} })) \\

    Entonces $(I \land  B) \implica wp(S, I)$\\



    \item $\{I \land B \land v_0 = fv\} S \{fv < v_0\} \equiv (I \land B \land v_0 = fv) \implica wp(S, fv < v_0)$

    Definimos la función variante como:
    $fv = |eventos| - i$\\ \\
    $wp(S,fv<v_0)\equiv wp(if\hspace{0.1cm} eventos[i]\hspace{0.1cm} Then\hspace{0.1cm} res = res*apuesta.c*pago.c\hspace{0.1cm} else\hspace{0.1cm} res*apuesta.s*pago.s, wp(i:= i + 1, |eventos|-i<v_0))$\\
    $wp(S,fv<v_0)\equiv wp(if\hspace{0.1cm} eventos[i]\hspace{0.1cm} Then\hspace{0.1cm} res = res*apuesta.c*pago.c\hspace{0.1cm} else\hspace{0.1cm} res*apuesta.s*pago.s,  |eventos|-(i+1)<v_0)$\\
    
    $wp(S,fv<v_0)\equiv def(eventos[i])\yLuego ((eventos[i]=True \land wp(S_1, |eventos|-i-1<v_0))\lor (eventos[i]=False \land wp(S_2, |eventos|-i-1<v_0))) 
    $ \\
    \\
    $wp(S_2,|eventos|-1-i<v_0) \equiv def(res*apuesta.s*pago.s)\yLuego |eventos|-1-i<v_0 \equiv |eventos|-1-i<v_0
    $
    \\
    \\
    $wp(S_1,|eventos|-1-i<v_0) \equiv def(res*apuesta.c*pago.c)\yLuego |eventos|-1-i<v_0 \equiv |eventos|-1-i<v_0
    $\\
    \\
    $wp(S,fv<v_0)\equiv 0\leq i <|eventos|\yLuego ((eventos[i]=True\land |eventos|-1-i<v_0)\lor (eventos[i]=False\land |eventos|-1-i<v_0))
    $\\
    \\
    $wp(S,fv<v_0)\equiv 0\leq i <|eventos| \yLuego |eventos|-1-i<v_0 
    $\\
    \\
    Ahora calculemos $I \land B \land v_0 = |eventos| - i$ y veamos si implica lo de arriba: \\
    A partir del Invariante tenemos que $0\leq i \leq |eventos| \implica 0\leq i < |eventos| $ \\Luego usando que $v_0 =|eventos|-i $  es verdadero se puede llegar a que $v_0 - 1 < v_0$ Lo cual es verdadero.\\
    \\
    Entonces $(I \land B \land v_0 = fv) \implica wp(S, fv < v_0)$
    
    \item $I \land fv \leq 0 \implica \neg B$\\
    A partir de que $fv\leq 0 \equiv |eventos|-i \leq 0 \equiv |eventos|\leq i \equiv \neg B$\\
    \newpage
    
           
\end{itemize}

\end{document}
